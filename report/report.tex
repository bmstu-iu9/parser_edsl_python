% !TeX TXS-program:bibliography = txs:///biber
\documentclass[14pt, russian]{scrartcl}
\let\counterwithout\relax
\let\counterwithin\relax
%\usepackage{lmodern}
\usepackage{float}
\usepackage{xcolor}
\usepackage{extsizes}
\usepackage{subfig}
\usepackage[export]{adjustbox}
\usepackage{tocvsec2} % возможность менять учитываемую глубину разделов в оглавлении
\usepackage[subfigure]{tocloft}
\usepackage[newfloat]{minted}
\removefromtoclist[float]{lol}
\captionsetup[listing]{position=top}

\AtBeginEnvironment{figure}{\vspace{0.5cm}}
\AtBeginEnvironment{table}{\vspace{0.5cm}}
\AtBeginEnvironment{listing}{\vspace{0.5cm}}
\AtBeginEnvironment{algorithm}{\vspace{0.5cm}}
\AtBeginEnvironment{minted}{\vspace{-0.5cm}}

\usepackage{fancyvrb}
\usepackage{ulem,bm,mathrsfs,ifsym} %зачеркивания, особо жирный стиль и RSFS начертание
\usepackage{sectsty} % переопределение стилей подразделов
%%%%%%%%%%%%%%%%%%%%%%%

%%% Поля и разметка страницы %%%
\usepackage{pdflscape}                              % Для включения альбомных страниц
\usepackage{geometry}                               % Для последующего задания полей
\geometry{a4paper,tmargin=2cm,bmargin=2cm,lmargin=3cm,rmargin=1cm} % тоже самое, но лучше

%%% Математические пакеты %%%
\usepackage{amsthm,amsfonts,amsmath,amssymb,amscd}  % Математические дополнения от AMS
\usepackage{mathtools}                              % Добавляет окружение multlined
\usepackage[perpage]{footmisc}
%\usepackage{times}

%%%% Установки для размера шрифта 14 pt %%%%
%% Формирование переменных и констант для сравнения (один раз для всех подключаемых файлов)%%
%% должно располагаться до вызова пакета fontspec или polyglossia, потому что они сбивают его работу
%\newlength{\curtextsize}
%\newlength{\bigtextsize}
%\setlength{\bigtextsize}{13pt}
\KOMAoptions{fontsize=14pt}

\makeatletter
\def\showfontsize{\f@size{} point}
\makeatother

%\makeatletter
%\show\f@size                                       % неплохо для отслеживания, но вызывает стопорение процесса, если документ компилируется без команды  -interaction=nonstopmode
%\setlength{\curtextsize}{\f@size pt}
%\makeatother

%шрифт times
\usepackage{tempora}
%\usepackage{pscyr}
%\setmainfont[Ligatures={TeX,Historic}]{Times New Roman}

   %%% Решение проблемы копирования текста в буфер кракозябрами
%    \input glyphtounicode.tex
%    \input glyphtounicode-cmr.tex %from pdfx package
%    \pdfgentounicode=1
    \usepackage{cmap}                               % Улучшенный поиск русских слов в полученном pdf-файле
    \usepackage[T1]{fontenc}                       % Поддержка русских букв
    \usepackage[utf8]{inputenc}                     % Кодировка utf8
    \usepackage[english, main=russian]{babel}            % Языки: русский, английский
%   \IfFileExists{pscyr.sty}{\usepackage{pscyr}}{}  % Красивые русские шрифты
%\renewcommand{\rmdefault}{ftm}
%%% Оформление абзацев %%%
\usepackage{indentfirst}                            % Красная строка
%\usepackage{eskdpz}

%%% Таблицы %%%
\usepackage{longtable}                              % Длинные таблицы
\usepackage{multirow,makecell,array}                % Улучшенное форматирование таблиц
\usepackage{booktabs}                               % Возможность оформления таблиц в классическом книжном стиле (при правильном использовании не противоречит ГОСТ)

%%% Общее форматирование
\usepackage{soulutf8}                               % Поддержка переносоустойчивых подчёркиваний и зачёркиваний
\usepackage{icomma}                                 % Запятая в десятичных дробях



%%% Изображения %%%
\usepackage{graphicx}                               % Подключаем пакет работы с графикой
\usepackage{wrapfig}

%%% Списки %%%
\usepackage{enumitem}

%%% Подписи %%%
\usepackage{caption}                                % Для управления подписями (рисунков и таблиц) % Может управлять номерами рисунков и таблиц с caption %Иногда может управлять заголовками в списках рисунков и таблиц
%% Использование:
%\begin{table}[h!]\ContinuedFloat - чтобы не переключать счетчик
%\captionsetup{labelformat=continued}% должен стоять до самого caption
%\caption{}
% либо ручками \caption*{Продолжение таблицы~\ref{...}.} :)

%%% Интервалы %%%
\addto\captionsrussian{%
  \renewcommand{\listingname}{Листинг}%
}
%%% Счётчики %%%
\usepackage[figure,table,section]{totalcount}               % Счётчик рисунков и таблиц
\DeclareTotalCounter{lstlisting}
\usepackage{totcount}                               % Пакет создания счётчиков на основе последнего номера подсчитываемого элемента (может требовать дважды компилировать документ)
\usepackage{totpages}                               % Счётчик страниц, совместимый с hyperref (ссылается на номер последней страницы). Желательно ставить последним пакетом в преамбуле

%%% Продвинутое управление групповыми ссылками (пока только формулами) %%%
%% Кодировки и шрифты %%%

%   \newfontfamily{\cyrillicfont}{Times New Roman}
%   \newfontfamily{\cyrillicfonttt}{CMU Typewriter Text}
	%\setmainfont{Times New Roman}
	%\newfontfamily\cyrillicfont{Times New Roman}
	%\setsansfont{Times New Roman}                    %% задаёт шрифт без засечек
%	\setmonofont{Liberation Mono}               %% задаёт моноширинный шрифт
%    \IfFileExists{pscyr.sty}{\renewcommand{\rmdefault}{ftm}}{}
%%% Интервалы %%%
%linespread-реализация ближе к реализации полуторного интервала в ворде.
%setspace реализация заточена под шрифты 10, 11, 12pt, под остальные кегли хуже, но всё же ближе к типографской классике.
\linespread{1.3}                    % Полуторный интервал (ГОСТ Р 7.0.11-2011, 5.3.6)
%\renewcommand{\@biblabel}[1]{#1}

%%% Гиперссылки %%%
\usepackage{hyperref}

%%% Выравнивание и переносы %%%
\sloppy                             % Избавляемся от переполнений
\clubpenalty=10000                  % Запрещаем разрыв страницы после первой строки абзаца
\widowpenalty=10000                 % Запрещаем разрыв страницы после последней строки абзаца

\makeatletter % малые заглавные, small caps shape
\let\@@scshape=\scshape
\renewcommand{\scshape}{%
  \ifnum\strcmp{\f@series}{bx}=\z@
    \usefont{T1}{cmr}{bx}{sc}%
  \else
    \ifnum\strcmp{\f@shape}{it}=\z@
      \fontshape{scsl}\selectfont
    \else
      \@@scshape
    \fi
  \fi}
\makeatother

%%% Подписи %%%
%\captionsetup{%
%singlelinecheck=off,                % Многострочные подписи, например у таблиц
%skip=2pt,                           % Вертикальная отбивка между подписью и содержимым рисунка или таблицы определяется ключом
%justification=centering,            % Центрирование подписей, заданных командой \caption
%}
%%%        Подключение пакетов                 %%%
\usepackage{ifthen}                 % добавляет ifthenelse
%%% Инициализирование переменных, не трогать!  %%%
\newcounter{intvl}
\newcounter{otstup}
\newcounter{contnumeq}
\newcounter{contnumfig}
\newcounter{contnumtab}
\newcounter{pgnum}
\newcounter{bibliosel}
\newcounter{chapstyle}
\newcounter{headingdelim}
\newcounter{headingalign}
\newcounter{headingsize}
\newcounter{tabcap}
\newcounter{tablaba}
\newcounter{tabtita}
%%%%%%%%%%%%%%%%%%%%%%%%%%%%%%%%%%%%%%%%%%%%%%%%%%

%%% Область упрощённого управления оформлением %%%

%% Интервал между заголовками и между заголовком и текстом
% Заголовки отделяют от текста сверху и снизу тремя интервалами (ГОСТ Р 7.0.11-2011, 5.3.5)
\setcounter{intvl}{3}               % Коэффициент кратности к размеру шрифта

%% Отступы у заголовков в тексте
\setcounter{otstup}{0}              % 0 --- без отступа; 1 --- абзацный отступ

%% Нумерация формул, таблиц и рисунков
\setcounter{contnumeq}{1}           % Нумерация формул: 0 --- пораздельно (во введении подряд, без номера раздела); 1 --- сквозная нумерация по всей диссертации
\setcounter{contnumfig}{1}          % Нумерация рисунков: 0 --- пораздельно (во введении подряд, без номера раздела); 1 --- сквозная нумерация по всей диссертации
\setcounter{contnumtab}{1}          % Нумерация таблиц: 0 --- пораздельно (во введении подряд, без номера раздела); 1 --- сквозная нумерация по всей диссертации

%% Оглавление
\setcounter{pgnum}{0}               % 0 --- номера страниц никак не обозначены; 1 --- Стр. над номерами страниц (дважды компилировать после изменения)

%% Библиография
\setcounter{bibliosel}{1}           % 0 --- встроенная реализация с загрузкой файла через движок bibtex8; 1 --- реализация пакетом biblatex через движок biber

%% Текст и форматирование заголовков
\setcounter{chapstyle}{1}           % 0 --- разделы только под номером; 1 --- разделы с названием "Глава" перед номером
\setcounter{headingdelim}{1}        % 0 --- номер отделен пропуском в 1em или \quad; 1 --- номера разделов и приложений отделены точкой с пробелом, подразделы пропуском без точки; 2 --- номера разделов, подразделов и приложений отделены точкой с пробелом.

%% Выравнивание заголовков в тексте
\setcounter{headingalign}{0}        % 0 --- по центру; 1 --- по левому краю

%% Размеры заголовков в тексте
\setcounter{headingsize}{0}         % 0 --- по ГОСТ, все всегда 14 пт; 1 --- пропорционально изменяющийся размер в зависимости от базового шрифта

%% Подпись таблиц
\setcounter{tabcap}{0}              % 0 --- по ГОСТ, номер таблицы и название разделены тире, выровнены по левому краю, при необходимости на нескольких строках; 1 --- подпись таблицы не по ГОСТ, на двух и более строках, дальнейшие настройки:
%Выравнивание первой строки, с подписью и номером
\setcounter{tablaba}{2}             % 0 --- по левому краю; 1 --- по центру; 2 --- по правому краю
%Выравнивание строк с самим названием таблицы
\setcounter{tabtita}{1}             % 0 --- по левому краю; 1 --- по центру; 2 --- по правому краю

%%% Рисунки %%%
\DeclareCaptionLabelSeparator*{emdash}{~--- }             % (ГОСТ 2.105, 4.3.1)
\captionsetup[figure]{labelsep=emdash,font=onehalfspacing,position=bottom}

%%% Таблицы %%%
\ifthenelse{\equal{\thetabcap}{0}}{%
    \newcommand{\tabcapalign}{\raggedright}  % по левому краю страницы или аналога parbox
}

\ifthenelse{\equal{\thetablaba}{0} \AND \equal{\thetabcap}{1}}{%
    \newcommand{\tabcapalign}{\raggedright}  % по левому краю страницы или аналога parbox
}

\ifthenelse{\equal{\thetablaba}{1} \AND \equal{\thetabcap}{1}}{%
    \newcommand{\tabcapalign}{\centering}    % по центру страницы или аналога parbox
}

\ifthenelse{\equal{\thetablaba}{2} \AND \equal{\thetabcap}{1}}{%
    \newcommand{\tabcapalign}{\raggedleft}   % по правому краю страницы или аналога parbox
}

\ifthenelse{\equal{\thetabtita}{0} \AND \equal{\thetabcap}{1}}{%
    \newcommand{\tabtitalign}{\raggedright}  % по левому краю страницы или аналога parbox
}

\ifthenelse{\equal{\thetabtita}{1} \AND \equal{\thetabcap}{1}}{%
    \newcommand{\tabtitalign}{\centering}    % по центру страницы или аналога parbox
}

\ifthenelse{\equal{\thetabtita}{2} \AND \equal{\thetabcap}{1}}{%
    \newcommand{\tabtitalign}{\raggedleft}   % по правому краю страницы или аналога parbox
}

\DeclareCaptionFormat{tablenocaption}{\tabcapalign #1\strut}        % Наименование таблицы отсутствует
\ifthenelse{\equal{\thetabcap}{0}}{%
    \DeclareCaptionFormat{tablecaption}{\tabcapalign #1#2#3}
    \captionsetup[table]{labelsep=emdash}                       % тире как разделитель идентификатора с номером от наименования
}{%
    \DeclareCaptionFormat{tablecaption}{\tabcapalign #1#2\par%  % Идентификатор таблицы на отдельной строке
        \tabtitalign{#3}}                                       % Наименование таблицы строкой ниже
    \captionsetup[table]{labelsep=space}                        % пробельный разделитель идентификатора с номером от наименования
}
\captionsetup[table]{format=tablecaption,singlelinecheck=off,font=onehalfspacing,position=top,skip=-5pt}  % многострочные наименования и прочее
\DeclareCaptionLabelFormat{continued}{Продолжение таблицы~#2}
\setlength{\belowcaptionskip}{.2cm}
\setlength{\intextsep}{0ex}

%%% Подписи подрисунков %%%
\renewcommand{\thesubfigure}{\asbuk{subfigure}}           % Буквенные номера подрисунков
\captionsetup[subfigure]{font={normalsize},               % Шрифт подписи названий подрисунков (не отличается от основного)
    labelformat=brace,                                    % Формат обозначения подрисунка
    justification=centering,                              % Выключка подписей (форматирование), один из вариантов
}
%\DeclareCaptionFont{font12pt}{\fontsize{12pt}{13pt}\selectfont} % объявляем шрифт 12pt для использования в подписях, тут же надо интерлиньяж объявлять, если не наследуется
%\captionsetup[subfigure]{font={font12pt}}                 % Шрифт подписи названий подрисунков (всегда 12pt)

%%% Настройки гиперссылок %%%

\definecolor{linkcolor}{rgb}{0.0,0,0}
\definecolor{citecolor}{rgb}{0,0.0,0}
\definecolor{urlcolor}{rgb}{0,0,0}

\hypersetup{
    linktocpage=true,           % ссылки с номера страницы в оглавлении, списке таблиц и списке рисунков
%    linktoc=all,                % both the section and page part are links
%    pdfpagelabels=false,        % set PDF page labels (true|false)
    plainpages=true,           % Forces page anchors to be named by the Arabic form  of the page number, rather than the formatted form
    colorlinks,                 % ссылки отображаются раскрашенным текстом, а не раскрашенным прямоугольником, вокруг текста
    linkcolor={linkcolor},      % цвет ссылок типа ref, eqref и подобных
    citecolor={citecolor},      % цвет ссылок-цитат
    urlcolor={urlcolor},        % цвет гиперссылок
    pdflang={ru},
}
\urlstyle{same}
%%% Шаблон %%%
%\DeclareRobustCommand{\todo}{\textcolor{red}}       % решаем проблему превращения названия цвета в результате \MakeUppercase, http://tex.stackexchange.com/a/187930/79756 , \DeclareRobustCommand protects \todo from expanding inside \MakeUppercase
\setlength{\parindent}{2.5em}                       % Абзацный отступ. Должен быть одинаковым по всему тексту и равен пяти знакам (ГОСТ Р 7.0.11-2011, 5.3.7).

%%% Списки %%%
% Используем дефис для ненумерованных списков (ГОСТ 2.105-95, 4.1.7)
%\renewcommand{\labelitemi}{\normalfont\bfseries~{---}}
\renewcommand{\labelitemi}{\bfseries~{---}}
\setlist{nosep,%                                    % Единый стиль для всех списков (пакет enumitem), без дополнительных интервалов.
    labelindent=\parindent,leftmargin=*%            % Каждый пункт, подпункт и перечисление записывают с абзацного отступа (ГОСТ 2.105-95, 4.1.8)
}
%%%%%%%%%%%%%%%%%%%%%%
%\usepackage{xltxtra} % load xunicode

\usepackage{ragged2e}
\usepackage[explicit]{titlesec}
\usepackage{placeins}
\usepackage{xparse}
\usepackage{csquotes}

\usepackage{listingsutf8}
\usepackage{url} %пакеты расширений
\usepackage{algorithm, algorithmicx}
\usepackage[noend]{algpseudocode}
\usepackage{blkarray}
\usepackage{chngcntr}
\usepackage{tabularx}
\usepackage[backend=biber,
    bibstyle=gost-numeric,
    citestyle=nature]{biblatex}
\newcommand*\template[1]{\text{<}#1\text{>}}
\addbibresource{biblio.bib}

\titleformat{name=\section,numberless}[block]{\normalfont\Large\centering}{}{0em}{#1}
\titleformat{\section}[block]{\normalfont\Large\bfseries\raggedright}{}{0em}{\thesection\hspace{0.25em}#1}
\titleformat{\subsection}[block]{\normalfont\Large\bfseries\raggedright}{}{0em}{\thesubsection\hspace{0.25em}#1}
\titleformat{\subsubsection}[block]{\normalfont\large\bfseries\raggedright}{}{0em}{\thesubsubsection\hspace{0.25em}#1}

\let\Algorithm\algorithm
\renewcommand\algorithm[1][]{\Algorithm[#1]\setstretch{1.5}}
%\renewcommand{\listingscaption}{Листинг}

\usepackage{pifont}
\usepackage{calc}
\usepackage{suffix}
\usepackage{csquotes}
\DeclareQuoteStyle{russian}
    {\guillemotleft}{\guillemotright}[0.025em]
    {\quotedblbase}{\textquotedblleft}
\ExecuteQuoteOptions{style=russian}
\newcommand{\enq}[1]{\enquote{#1}}
\newcommand{\eng}[1]{\begin{english}#1\end{english}}
% Подчиненные счетчики в окружениях http://old.kpfu.ru/journals/izv_vuz/arch/sample1251.tex
\newcounter{cTheorem}
\newcounter{cDefinition}
\newcounter{cConsequent}
\newcounter{cExample}
\newcounter{cLemma}
\newcounter{cConjecture}
\newtheorem{Theorem}{Теорема}[cTheorem]
\newtheorem{Definition}{Определение}[cDefinition]
\newtheorem{Consequent}{Следствие}[cConsequent]
\newtheorem{Example}{Пример}[cExample]
\newtheorem{Lemma}{Лемма}[cLemma]
\newtheorem{Conjecture}{Гипотеза}[cConjecture]

\renewcommand{\theTheorem}{\arabic{Theorem}}
\renewcommand{\theDefinition}{\arabic{Definition}}
\renewcommand{\theConsequent}{\arabic{Consequent}}
\renewcommand{\theExample}{\arabic{Example}}
\renewcommand{\theLemma}{\arabic{Lemma}}
\renewcommand{\theConjecture}{\arabic{Conjecture}}
%\makeatletter
\NewDocumentCommand{\Newline}{}{\text{\\}}
\newcommand{\sequence}[2]{\ensuremath \left(#1,\ \dots,\ #2\right)}

\definecolor{mygreen}{rgb}{0,0.6,0}
\definecolor{mygray}{rgb}{0.5,0.5,0.5}
\definecolor{mymauve}{rgb}{0.58,0,0.82}
\renewcommand{\listalgorithmname}{Список алгоритмов}
\floatname{algorithm}{Листинг}
\renewcommand{\lstlistingname}{Листинг}
\renewcommand{\thealgorithm}{\arabic{algorithm}}

\newcommand{\refAlgo}[1]{(листинг \ref{#1})}
\newcommand{\refImage}[1]{(рисунок \ref{#1})}

\renewcommand{\theenumi}{\arabic{enumi}.}% Меняем везде перечисления на цифра.цифра
\renewcommand{\labelenumi}{\arabic{enumi}.}% Меняем везде перечисления на цифра.цифра
\renewcommand{\theenumii}{\arabic{enumii}}% Меняем везде перечисления на цифра.цифра
\renewcommand{\labelenumii}{(\arabic{enumii})}% Меняем везде перечисления на цифра.цифра
\renewcommand{\theenumiii}{\roman{enumiii}}% Меняем везде перечисления на цифра.цифра
\renewcommand{\labelenumiii}{(\roman{enumiii})}% Меняем везде перечисления на цифра.цифра
%\newfontfamily\AnkaCoder[Path=src/fonts/]{AnkaCoder-r.ttf}
\renewcommand{\labelitemi}{---}
\renewcommand{\labelitemii}{---}

%\usepackage{courier}

\lstdefinelanguage{Refal}{
  alsodigit = {.,<,>},
  morekeywords = [1]{$ENTRY},
  morekeywords = [2]{Go, Put, Get, Open, Close, Arg, Add, Sub, Mul, Div, Symb, Explode, Implode},
  %keyword4
  morekeywords = [3]{<,>},
  %keyword5
  morekeywords = [4]{e.,t.,s.},
  sensitive = true,
  morecomment = [l]{*},
  morecomment = [s]{/*}{*/},
  commentstyle = \color{mygreen},
  morestring = [b]",
  morestring = [b]',
  stringstyle = \color{purple}
}

\makeatletter
\def\p@subsection{}
\def\p@subsubsection{\thesection\,\thesubsection\,}
\makeatother
\newcommand{\prog}[1]{{\ttfamily\small#1}}
\lstset{ %
  backgroundcolor=\color{white},   % choose the background color; you must add \usepackage{color} or \usepackage{xcolor}
  basicstyle=\ttfamily\footnotesize,
  %basicstyle=\footnotesize\AnkaCoder,        % the size of the fonts that are used for the code
  breakatwhitespace=false,         % sets if automatic breaks shoulbd only happen at whitespace
  breaklines=true,                 % sets automatic line breaking
  captionpos=top,                    % sets the caption-position to bottom
  commentstyle=\color{mygreen},    % comment style
  deletekeywords={...},            % if you want to delete keywords from the given language
  escapeinside={\%*}{*)},          % if you want to add LaTeX within your code
  extendedchars=true,              % lets you use non-ASCII characters; for 8-bits encodings only, does not work with UTF-8
  inputencoding=utf8,
  frame=single,                    % adds a frame around the code
  keepspaces=true,                 % keeps spaces in text, useful for keeping indentation of code (possibly needs columns=flexible)
  keywordstyle=\bf,       % keyword style
  language=Refal,                    % the language of the code
  morekeywords={<,>,$ENTRY,Go,Arg, Open, Close, e., s., t., Get, Put},
  							       % if you want to add more keywords to the set
  numbers=left,                    % where to put the line-numbers; possible values are (none, left, right)
  numbersep=5pt,                   % how far the line-numbers are from the code
  xleftmargin=25pt,
  xrightmargin=25pt,
  numberstyle=\small\color{black}, % the style that is used for the line-numbers
  rulecolor=\color{black},         % if not set, the frame-color may be changed on line-breaks within not-black text (e.g. comments (green here))
  showspaces=false,                % show spaces everywhere adding particular underscores; it overrides 'showstringspaces'
  showstringspaces=false,          % underline spaces within strings only
  showtabs=false,                  % show tabs within strings adding particular underscores
  stepnumber=1,                    % the step between two line-numbers. If it's 1, each line will be numbered
  stringstyle=\color{mymauve},     % string literal style
  tabsize=8,                       % sets default tabsize to 8 spaces
  title=\lstname                   % show the filename of files included with \lstinputlisting; also try caption instead of title
}
\newcommand{\anonsection}[1]{\cleardoublepage
\phantomsection
\addcontentsline{toc}{section}{\protect\numberline{}#1}
\section*{#1}\vspace*{2.5ex} % По госту положены 3 пустые строки после заголовка ненумеруемого раздела
}
\newcommand{\sectionbreak}{\clearpage}
\renewcommand{\sectionfont}{\normalsize} % Сбиваем стиль оглавления в стандартный
\renewcommand{\cftsecleader}{\cftdotfill{\cftdotsep}} % Точки в оглавлении напротив разделов

\renewcommand{\cftsecfont}{\normalfont\large} % Переключение на times в содержании
\renewcommand{\cftsubsecfont}{\normalfont\large} % Переключение на times в содержании

\usepackage{caption}
%\captionsetup[table]{justification=raggedleft}
%\captionsetup[figure]{justification=centering,labelsep=endash}
\usepackage{amsmath}    % \bar    (матрицы и проч. ...)
\usepackage{amsfonts}   % \mathbb (символ для множества действительных чисел и проч. ...)
\usepackage{mathtools}  % \abs, \norm
    \DeclarePairedDelimiter\abs{\lvert}{\rvert} % операция модуля
    \DeclarePairedDelimiter\norm{\lVert}{\rVert} % операция нормы
\DeclareTextCommandDefault{\textvisiblespace}{%
  \mbox{\kern.06em\vrule \@height.3ex}%
  \vbox{\hrule \@width.3em}%
  \hbox{\vrule \@height.3ex}}
\newsavebox{\spacebox}
\begin{lrbox}{\spacebox}
\verb*! !
\end{lrbox}
\newcommand{\aspace}{\usebox{\spacebox}}
\DeclareTotalCounter{listing}

\makeatletter
\renewcommand*{\p@subsubsection}{}
\makeatother

\begin{document}
\sloppy

\def\figurename{Рисунок}

\begin{titlepage}
	\thispagestyle{empty}
	\newpage

	\vspace*{-30pt}
	\hspace{-45pt}
	\begin{minipage}{0.17\textwidth}
		\hspace*{-20pt}\centering
		\includegraphics[width=1.3\textwidth]{images/emblem.png}
	\end{minipage}
	\begin{minipage}{0.82\textwidth}\small \textbf{
			\vspace*{-0.7ex}
			\hspace*{-10pt}\centerline{Министерство науки и высшего образования Российской Федерации}
			\vspace*{-0.7ex}
			\centerline{Федеральное государственное бюджетное образовательное учреждение }
			\vspace*{-0.7ex}
			\centerline{высшего образования}
			\vspace*{-0.7ex}
			\centerline{<<Московский государственный технический университет}
			\vspace*{-0.7ex}
			\centerline{имени Н.Э. Баумана}
			\vspace*{-0.7ex}
			\centerline{(национальный исследовательский университет)>>}
			\vspace*{-0.7ex}
			\centerline{(МГТУ им. Н.Э. Баумана)}}
	\end{minipage}

	\vspace{-2pt}
	\hspace{-34.5pt}\rule{\textwidth}{2.5pt}

	\vspace*{-20.3pt}
	\hspace{-34.5pt}\rule{\textwidth}{0.4pt}

	\vspace{0.5ex}
	\noindent \small ФАКУЛЬТЕТ\hspace{80pt} <<Информатика и системы управления>>

	\vspace*{-16pt}
	\hspace{35pt}\rule{0.855\textwidth}{0.4pt}

	\vspace{0.5ex}
	\noindent \small КАФЕДРА\hspace{50pt} <<Теоретическая информатика и компьютерные технологии>>

	\vspace*{-16pt}
	\hspace{25pt}\rule{0.875\textwidth}{0.4pt}


	\vspace{3em}

	\begin{center}
		\Large \bf{РАСЧЕТНО-ПОЯСНИТЕЛЬНАЯ ЗАПИСКА\\\textbf{\textit{К КУРСОВОЙ РАБОТЕ\\НА ТЕМУ:}} \\}
	\end{center}

	\vspace*{-6ex}
	\begin{center}
		\Large{\textit{\textbf{<<Расширение возможностей}}}

		\vspace*{-3ex}
		\rule{0.9\textwidth}{1.2pt}

		\vspace*{-0.2ex}
		\Large{\textit{\textbf{библиотеки parser\_edsl>>}}}
		\vspace*{-3ex}
		\vspace*{-0.2ex}
		\rule{0.9\textwidth}{1.2pt}

		\vspace*{-0.2ex}
		\rule{0.9\textwidth}{1.2pt}

		\vspace*{-0.2ex}
		\rule{0.9\textwidth}{1.2pt}

		\vspace*{-0.2ex}
		\rule{0.9\textwidth}{1.2pt}
	\end{center}

	\vspace{\fill}


	\newlength{\ML}
	\settowidth{\ML}{«\underline{\hspace{0.7cm}}» \underline{\hspace{2cm}}}

	\noindent Студент \underline{\text{ИУ9-71Б}} \hfill \underline{ \hspace{4cm}}\quad
	\underline{\parbox{4cm}{\centering\raisebox{0.25ex}{ Гречко Г.В.}}}

	\vspace{-2.1ex}
	\noindent\hspace{9ex}\scriptsize{(Группа)}\normalsize\hspace{170pt}\hspace{2ex}\scriptsize{(Подпись, дата)}\normalsize\hspace{30pt}\hspace{6ex}\scriptsize{(И.О. Фамилия)}\normalsize

	\bigskip

	\noindent Руководитель  \hfill \underline{\hspace{4cm}}\quad
	\underline{\parbox{4cm}{\centering\raisebox{0.25ex}{ Коновалов А.В.}}}

	\vspace{-2ex}
	\noindent\hspace{13.5ex}\normalsize\hspace{170pt}\hspace{2ex}\scriptsize{(Подпись, дата)}\normalsize\hspace{30pt}\hspace{6ex}\scriptsize{(И.О. Фамилия)}\normalsize

	\bigskip

	\noindent Консультант\hfill \underline{\hspace{4cm}}\quad
	\underline{\hspace{4cm}}

	\vspace{-2ex}
	\noindent\hspace{13.5ex}\normalsize\hspace{170pt}\hspace{2ex}\scriptsize{(Подпись, дата)}\normalsize\hspace{30pt}\hspace{6ex}\scriptsize{(И.О. Фамилия)}\normalsize
	\vfill

	%\vspace{\fill}



	\begin{center}
		\textsl{2025 г.}
	\end{center}
\end{titlepage}

%\renewcommand{\ttdefault}{pcr}

\setlength{\tabcolsep}{3pt}
\newpage
\setcounter{page}{2}
%----------------------------------------------------------------------------
%                  ОТСЮДА --- СОБСТВЕННО ТЕКСТ
%----------------------------------------------------------------------------

\newpage
\renewcommand\contentsname{\hfill{\normalfont{СОДЕРЖАНИЕ}}\hfill}  %Оглавление
\tableofcontents
\newpage
\anonsection{Введение}  %Введение

В современном мире разработка компиляторов и интерпретаторов занимает важное место в компьютерных науках, позволяя
создавать эффективные инструменты для анализа и трансляции языков программирования. Для упрощения разработки
компиляторов и повышения их
гибкости широко применяются специализированные библиотеки, реализующие различные методы синтаксического анализа.
Библиотека parser\_edsl является одним из таких инструментов, предоставляющих возможность описания синтаксиса
языков, используя средства объектно-ориентированных языков программирования (python в случае данной библиотеки).

Цель данной курсовой работы заключается в расширении возможностей библиотеки parser\_edsl
за счёт реализации дополнительной функциональности:

\begin{itemize}
	\item Поддержка лексических ошибок --- функции вычисления атрибутов токенов
	могут возбуждать библиотечное исключение, которое затем транслируется
	в синтаксическую ошибку.
	\item Поддержка режимов разбора <<предсказывающий анализ>> (только для LL(1)-
	грамматик) и <<алгоритм Эрли>>.
	\item Возможность указания приоритета и ассоциативности в стиле Bison' a.
\end{itemize}

% \newpage

\section{Основные теоретические сведения}

\subsection{Компилирование}

Компиляция — это процесс преобразования исходного кода, написанного на одном языке программирования,
на другой язык или машинный код,
обычно пригодный для выполнения целевой машиной. Основная цель компиляции заключается в том, чтобы
трансформировать высокоуровневые конструкции, понятные человеку, в низкоуровневые инструкции, исполняемые
процессором. Помимо этого, компиляторы могут выполнять различные оптимизации или находить ошибки в коде программы.

Стандартный процесс компиляции обычно включает несколько последовательных фаз, каждая из которых выполняет свою задачу:

\begin{enumerate}
	\item \textbf{Лексический анализ (токенизация)} --- На этой фазе исходный текст программы разбивается на
	элементарные единицы — лексемы (токены), такие как ключевые слова, идентификаторы, операторы и литералы.
	Лексический анализатор также занимается удалением незначащих символов (например, пробелов, комментариев)
	и обработкой лексических ошибок.

	\item \textbf{Синтаксический анализ} --- После лексического анализа последовательность токенов анализируется
	с точки зрения синтаксиса языка. В ходе синтаксического разбора строится синтаксическое дерево,
	которое отражает иерархическую структуру программы согласно грамматике языка. Здесь выявляются синтаксические
	ошибки, если входной поток токенов не соответствует правилам языка.

	\item \textbf{Семантический анализ} --- Построенное синтаксическое дерево подвергается дополнительной проверке
	на семантическую корректность. На этой фазе осуществляется анализ типов, разрешение идентификаторов,
	проверка областей видимости и другие аспекты, не охватываемые синтаксическим анализом. Семантический анализ
	позволяет выявить логические и концептуальные ошибки, которые не были обнаружены ранее.

	\item \textbf{Генерация промежуточного кода} --- Для упрощения дальнейших оптимизаций исходный код преобразуется
	в промежуточное представление (IR), которое, как правило, является независимым от целевой аппаратной платформы.
	Такое представление облегчает проведение оптимизаций и последующую трансляцию в машинный код.

	\item \textbf{Оптимизация} --- На этой фазе производится улучшение промежуточного представления для повышения
	эффективности программы. Оптимизации могут включать устранение избыточных вычислений, оптимизацию циклов,
	минимизацию использования памяти и прочие техники, направленные на улучшение производительности.

	\item \textbf{Генерация целевого кода} --- Заключительный этап компиляции заключается в преобразовании
	оптимизированного промежуточного представления в конечный код, который может быть исполнен на конкретной
	аппаратной платформе. Это может быть машинный код, ассемблер или другой специализированный формат.
\end{enumerate}

Каждый этап решает специфическую задачу, что облегчает обнаружение ошибок и их последующее исправление, а
также позволяет проводить конкретные оптимизации, направленные на ускорение одного из этапов компиляции.
В результате получается надёжный и эффективный компилятор, пригодный для решения серьезных задач.

Библиотека parser\_edsl предоставляет инструменты для решения задач лексического, синтаксического и семантического
анализов, поэтому рассмотрим эти шаги подробнее.

\subsection{Лексический разбор}

Лексический анализатор читает текст из входного потока и разбивает его на лексемы. Если разбить на лексемы не получится,
анализатор вернет ошибку лексического разбора. Псевдокод для лексического анализатора, используемого в библиотеке
parser\_edsl, представлен на листинге \ref{lst:lex}.

\begin{listing}[H]
	\caption{Лексический разбор}
	\label{lst:lex}
\begin{algorithm}[H]
\begin{algorithmic}[1]
\Procedure{NextToken}{domains, text, skip\_token}
  \State $pos \gets \text{начальная позиция (объект Position)}$
  \While{$pos.offset < \text{length}(text)$}
    \State $offset \gets pos.offset$
    \State $matches \gets \varnothing$
    \ForAll{доменов $d$ из \texttt{domains}}
      \State $(length, attr) \gets d.match(text, offset)$
      \State $matches \gets matches \cup \{(d, d.priority, length, attr)\}$
    \EndFor
    \State $(domain, priority, length, attr) \gets \max(matches,\ \lambda (d, pr, l, a): (l, pr))$
    \State \textbf{assert} $length > 0$
    \If{$attr$ соответствует \texttt{ErrorTerminal}}
      \State \textbf{raise} LexerError($pos$, $text$)
    \EndIf
    \State $new\_pos \gets pos.\text{shift}(text[offset : offset + length])$
    \State $fragment \gets \text{Fragment}(pos, new\_pos)$
    \State $pos \gets new\_pos$
    \If{$attr \neq skip\_token$}
      \State \Return Token($domain$, $fragment$, $attr$)
    \EndIf
  \EndWhile
  \State \Return Token(EOF\_SYMBOL, Fragment($pos$, $pos$), \texttt{None})
\EndProcedure
\end{algorithmic}
\end{algorithm}
\end{listing}

\subsection{Алгоритмы синтаксического разбора}

Существует множество алгоритмов синтаксического разбора, каждый из которых имеет свои преимущества и ограничения.
Выбор конкретного метода определяется характеристиками грамматики языка и требованиями к скорости,
объёму памяти и способности обработки неоднозначностей. В данной работе будут затрагиваться алгоритмы LR разбора,
Эрли и предсказывающий разбор для LL(1) грамматик. Рассмотрим их подробнее.

\subsubsection{LR разбор}

LR-разбор (от англ. Left-to-right, Rightmost derivation in reverse) является одним из самых мощных и широко
используемых методов синтаксического анализа. Основная идея метода заключается в последовательном сканировании
входной строки слева направо с использованием стека для хранения промежуточных состояний. Алгоритм выполняет
так называемые операции сдвига (shift) и свёртки (reduce), преобразуя последовательность токенов в структуру,
соответствующую правилам грамматики.

К преимуществам LR-разбора относятся:

\begin{itemize}
	\item Обработка широкого класса грамматик --- алгоритм способен работать с грамматиками, для которых невозможно
	применить более простые виды анализаторов (например --- LL(1) предсказывающий разбор).
	\item Детерминированность --- При корректном построении таблиц переходов алгоритм не требует отката, что
	обеспечивает высокую скорость анализа.
\end{itemize}

Однако, работа LR-анализаторов может оказаться достаточно затратной по вычислительным ресурсам для больших грамматик,
поскольку таблицы переходов могут иметь значительные размеры, что накладывает ограничения на использование
данного алгоритма.

На листинге \ref{lst:lr} представлен псевдокод алгоритма LR-разбора.

\begin{listing}[H]
	\caption{LR-разбор}
	\label{lst:lr}
\begin{algorithm}[H]
	\begin{algorithmic}[1]
	\Procedure{LRParse}{tokens, ACTION, GOTO}
	  \State $stack \gets [0]$
	  \State $i \gets 0$
	  \While{true}
		\State $s \gets$ \textbf{top}$(stack)$
		\State $a \gets tokens[i]$
		\State $action \gets$ ACTION[$s, a$]
		\If{$action$ имеет вид <<shift $s'$>>}
		  \State \textbf{push}($s'$) на $stack$
		  \State $i \gets i + 1$
		\ElsIf{$action$ имеет вид <<reduce $A \to \beta$>>}
		  \State \textbf{pop} $2\cdot |\beta|$ элементов из $stack$
		  \State $t \gets$ \textbf{top}$(stack)$
		  \State \textbf{push}(GOTO[$t, A$]) на $stack$
		\ElsIf{$action$ является <<accept>>}
		  \State \Return ok
		\Else
		  \State \Return fail
		\EndIf
	  \EndWhile
	\EndProcedure
	\end{algorithmic}
	\end{algorithm}
\end{listing}

\subsubsection{Алгоритм Эрли}

\label{sec:earleyint}

Алгоритм Эрли представляет собой универсальный метод синтаксического анализа, способный обрабатывать любые
контекстно-свободные грамматики, включая неоднозначные. Данный алгоритм основан на принципах динамического
программирования и допускает параллельное рассмотрение нескольких вариантов разбора входной строки.

Главное преимущество алгоритма Эрли --- способность работать с грамматиками, содержащими неоднозначности.

Основным недостатком алгоритма является его потенциальная вычислительная сложность. В худших случаях алгоритм
может потребовать значительных вычислительных ресурсов, что так же, как и в случае LR разбора, накладывает
дополнительные ограничения для практического применения.


На листинге \ref{lst:earley} представлен псевдокод алгоритма Эрли.


\begin{listing}[H]
	\caption{Алгоритм Эрли}
	\label{lst:earley}
\begin{algorithm}[H]
	\begin{algorithmic}[1]
	\Procedure{EarleyParse}{tokens, Grammar}
	  \State $n \gets$ длина(tokens)
	  \State Инициализировать массив \texttt{chart[0..n]}, где каждая ячейка содержит множество состояний
	  \State Добавить в \texttt{chart[0]} все состояния вида $(S \to \cdot \alpha, 0)$ для стартового символа $S$
	  \For{$k \gets 0$ \textbf{to} $n$}
		\Repeat
		  \ForAll{состояний $(A \to \alpha \cdot B \beta, j)$ в \texttt{chart[k]}}
			\If{$B$ --- нетерминал}
			  \State \textbf{Predictor:} Для каждого правила $B \to \gamma$ добавить состояние $(B \to \cdot \gamma, k)$ в \texttt{chart[k]}
			\EndIf
		  \EndFor
		\Until{не появляются новые состояния в \texttt{chart[k]}}

		\ForAll{состояний $(A \to \alpha \cdot a \beta, j)$ в \texttt{chart[k]}}
		  \If{$a = tokens[k]$}
			\State \textbf{Scanner:} Добавить состояние $(A \to \alpha a \cdot \beta, j)$ в \texttt{chart[k+1]}
		  \EndIf
		\EndFor

		\Repeat
		  \ForAll{состояний $(B \to \gamma \cdot, j)$ в \texttt{chart[k]}}
			\ForAll{состояний $(A \to \alpha \cdot B \beta, i)$ в \texttt{chart[j]}}
			  \State \textbf{Completer:} Добавить состояние $(A \to \alpha B \cdot \beta, i)$ в \texttt{chart[k]}
			\EndFor
		  \EndFor
		\Until{не появляются новые состояния в \texttt{chart[k]}}
	  \EndFor
	  \State \Return проверка наличия в \texttt{chart[n]} состояния $(S \to \gamma \cdot, 0)$
	\EndProcedure
	\end{algorithmic}
	\end{algorithm}
\end{listing}


\subsubsection{Предсказывающий разбор для LL(1) грамматик}

\label{sec:ll1int}

Предсказывающий разбор является одним из наиболее интуитивно понятных и простых методов синтаксического анализа.
Он применяется для грамматик, удовлетворяющих условию LL(1) – то есть таких, для которых по текущему входному
символу и состоянию анализатора можно однозначно определить, какое правило применять.

Преимуществами данного метода являются простота в его реализации и высокая скорость работы и, благодаря
однозначному выбору следующего шага, обеспечивается высокая скорость разбора.

На листинге \ref{lst:ll1} представлен псевдокод алгоритма предсказывающего ll1 разбора.

\begin{listing}
	\caption{Предсказывающий разбор для LL(1)-грамматик}
	\label{lst:ll1}
\begin{algorithm}[H]
	\begin{algorithmic}[1]
	\Procedure{LL1Parse}{tokens, table, Start}
	  \State $stack \gets [\$, Start]$ \Comment{Знак \$ обозначает конец ввода}
	  \State $i \gets 0$
	  \While{$stack$ не пуст}
		\State $X \gets$ \textbf{top}$(stack)$
		\State $a \gets tokens[i]$
		\If{$X$ --- терминальный символ или \$}
		  \If{$X = a$}
			 \State \textbf{pop}($stack$)
			 \State $i \gets i + 1$
		  \Else
			 \State \Return ошибка разбора
		  \EndIf
		\Else
		  \If{table[$X, a$] определена как правило $X \to Y_1 Y_2 \dots Y_k$}
			 \State \textbf{pop}($stack$)
			 \State \textbf{push}($Y_k, Y_{k-1}, \dots, Y_1$) на $stack$
		  \Else
			 \State \Return fail
		  \EndIf
		\EndIf
	  \EndWhile
	  \State \Return ok
	\EndProcedure
	\end{algorithmic}
\end{algorithm}
\end{listing}

\subsection{Семантический разбор}

Семантический разбор, или семантический анализ, является важным этапом компиляции, который следует
после синтаксического анализа. Его основная задача заключается в проверке корректности программы с
точки зрения смысловых правил языка, которые не могут быть выражены исключительно с помощью синтаксиса.

К основным задачам семантического разбора относятся:

\begin{itemize}
	\item Проверка типов --- проверка совместимости типов данных в выражениях и операциях
	\item Проверка объявления идентификаторов
	\item Проверка областей видимости --- проверка, что , что переменные и другие идентификаторы используются в
	рамках своих областей видимости, предотвращая возможные конфликты имён или попытки обращения к
	несуществующим объектам
	\item Проверка корректности использования операций --- например, проверка деления на 0
\end{itemize}

Семантический разбор обычно основывается на специальной символической таблице, которая формируется и
обновляется в процессе обработки исходного кода. Эта таблица содержит сведения о типах данных,
объявленных переменных, функциях, классах и других конструкциях языка. При анализе каждой новой
конструкции компилятор проверяет её корректность, используя информацию из таблицы символов, что
позволяет обнаруживать ошибки, не выявляемые на этапе синтаксического анализа.

\subsection{Ассоциативность и приоритет операций на примере Bison}

В системах генерации парсеров, таких как Bison, одной из проблем является разрешение неоднозначностей,
возникающих при разборе арифметических и логических выражений, где один и тот же набор токенов может
соответствовать различным структурам дерева разбора. Для решения этой проблемы Bison предоставляет механизм
задания ассоциативности и приоритета операторов.

В Bison приоритеты и ассоциативность операторов определяются с помощью директив, таких как
\verb|%left|, \verb|%right| и \verb|%nonassoc|. Эти директивы позволяют:

\begin{itemize}
	\item \textbf{Установить ассоциативность:}
	\begin{itemize}
		 \item \verb|%left| определяет левую ассоциативность, что означает, что цепочка операторов будет
		 группироваться слева. Например, выражение \verb|a - b - c| интерпретируется как \verb|(a - b) - c|.
		 \item \verb|%right| задаёт правую ассоциативность, что приводит к группировке справа, как в случае
		 выражения \verb|a ^ b ^ c|, которое будет разобрано как \verb|a ^ (b ^ c)|.
		 \item \verb|%nonassoc| используется для операторов, для которых не допускается последовательное
		 применение без явного указания порядка, что помогает избегать конфликтов.
	\end{itemize}
	\item \textbf{Определить приоритет:} Операторы, объявленные позже, автоматически получают более высокий
	приоритет, чем операторы, указанные ранее. Таким образом, если в грамматике присутствуют, например,
	как операторы сложения и вычитания, так и операторы умножения и деления, то умножение и деление будут
	иметь более высокий приоритет, что позволяет корректно интерпретировать выражение \verb|a + b * c| как
	\verb|a + (b * c)|.

\end{itemize}

Пример объявления операторов в Bison может выглядеть следующим образом:

\begin{verbatim}
	 %left '+' '-'
	 %left '*' '/'
	 %right '^'
\end{verbatim}

В данном примере:
\begin{itemize}
	\item Операторы \verb|+| и \verb|-| обладают одинаковой (левой)
	ассоциативностью и более низким приоритетом.
	\item Операторы \verb|*| и \verb|/| также левосторонние,
	но имеют более высокий приоритет по сравнению с \verb|+| и \verb|-|.
	\item Оператор \verb|^| обладает
	наивысшим приоритетом и является правоассоциативным.
\end{itemize}

\subsection{Технический обзор библиотеки parser\_edsl}

parser\_edsl – это библиотека на Python, предназначенная для описания грамматики языка и построения парсеров для них.
 Основная идея заключается в использовании Embedded Domain Specific Language (EDSL)
для определения лексических доменов посредством регулярных выражений и синтаксиса грамматики в стиле, похожем
на Бэкусову нормальную форму (BNF). Такая реализация позволяет в исходном тексте программы задавать правила
языка в декларативном виде, что упрощает процесс построения компилятора для учебных и исследовательских целей.

Основные компоненты библиотеки включают в себя:

\begin{itemize}
	\item Инструменты для описания грамматики --- такие классы, как \texttt{Terminal}, \texttt{NonTerminal}, позволяют легко и понятно описывать правила грамматики.
		Также, для класса\texttt{NonTerminal} поддержана перегрузка оператора \texttt{|=}, которая позволяет не только описывать
		правило, но и добавить семантическое действие(свертку), которое определяет, как будут
		вычисляться атрибуты найденных элементов.
	\item Класс \texttt{Lexer}, реализующий разбиение входного потока на лексемы.
	\item Класс \texttt{Parser}, реализующий алгоритм LR-разбора.  Библиотека строит LR(0)-автомат, реализует
	функции closure и goto, а затем формирует таблицу разбора (представлена классом \texttt{ParsingTable}).
	\item Для информирования пользователя о возникших ошибках при синтаксическом анализе реализован класс \texttt{ParseError}.
	 При обнаружении неожидаемого символа парсер генерирует исключение с подробным сообщением, в
	 котором указываются позиция ошибки и символы, которые ожидались на вход.
\end{itemize}

Также, библиотека написана с объектно-ориентированным подходом, то есть структурирована в виде набора классов,
что способствует модульности и удобству расширения. Каждый класс содержит только свою функциональность
(например, лексический анализ, описание грамматики, синтаксический анализ и обработка ошибок).

\section{Разработка расширений для библиотеки}

\subsection{Планируемые изменения в библиотеку}

Основная цель работы заключается в доработке функциональности библиотеки за счёт добавления поддержки лексических
ошибок, внедрения альтернативных режимов синтаксического разбора (предсказывающий анализ для LL(1)-грамматик
и алгоритм Эрли), а также реализации механизма задания приоритета и ассоциативности в стиле Bison. В данном разделе
рассмотрим подробно, что потребуется реализовать в каждом из пунктов технического задания.

\subsubsection{Поддержка ошибки вычисления атрибутов}

Аттрибуты у токенов --- важная часть в работе компилятора. Например, благодаря этому можно находить такие ошибки, как
деление на 0 и сообщать о них пользователю до выполнения программы. Такие ошибки можно находить и обрабатывать
на стадии семантического анализа, но что делать, если атрибут вычислить не удается?

Например, у нас идет разбор
выражения семантического версионирования SemVer. Если проверку на формат и используемые символы можно сделать с помощью
регулярного выражения, то что делать, если, например, версия должна быть в определенном числовом диапазоне? То есть
вроде токен и распознался лексическим анализатором, но вычислить его атрибут не получится. В таком случае будет
крайне полезным выбросить исключение, говорящее об этом, которое парсер правильно обработает и пробросит наверх.

Однако, сейчас в библиотеке есть два типа исключений: \texttt{LexerError} и \texttt{ParseError}, интерфейс которых
приведен на листинге \ref{lst:curerrros}. Как видно, их интерфейс не позволяет нативно передать информацию об ошибке
вычисления атрибута.

\begin{listing}[H]
	\caption{Интерфейс исключений \texttt{LexerError} и \texttt{ParseError}.}
	\label{lst:curerrros}
	\inputminted[style=bw, frame=single,fontsize = \footnotesize, linenos=false, xleftmargin = 1.5em]{python}{./listings/curerrros.py}
\end{listing}

Следовательно, необходимо разработать новый класс ошибок \texttt{TokenAttributeError}, конструктор которого будет принимать
сообщение об ошибке и обрабатываться программой для получения пользователем понятной ошибки с указанием позиции,
где произошла исключительная ситуация. Программа должна будет обернуть исключение в LexerError и потом обернуть его
в ParseError исключение, которое уже должно будет обрабатываться пользователем.

Также, для улучшения пользовательского
опыта необходимо предусмотреть в классе ParseError переопределение стандартного сообщения об ошибке, чтобы
сообщение для пользователя выглядело более лаконично и понятно.
Например, ошибка может выглядеть так:
\texttt{Ошибка (1, 4): Слишком большое значение: 300}.

Для тестирования данной функциональности предлагается использовать библиотеку pytest, так как в ней есть все необходимые
инструменты для проверки, что программа выбрасывает исключение и что именно это за исключение.

\subsubsection{Разработка предсказывающего разбора}

Принцип работы и особенности данного алгоритма разбора были описаны в разделе \ref{sec:ll1int}. Попробуем оценить,
как лучше этот алгоритм реализовать на основе уже существующей кодовой базы библиотеки, чтобы интегрировать его
максимально удобно и не вызывая поломок в других частях библиотеки.

В текущей версии \texttt{parser\_edsl} уже реализована хорошая структура кода, отделяющая различные этапы работы
библиотеки. В реализации нового алгоритма стоит выделить следующие компоненты:

\begin{itemize}
	\item \textbf{ParseTreeNode} --- узел дерева разбора для LL(1). Узел должен в себе хранить терминал/нетерминал
	и информацию о них(Token --- в случае терминала и список дочерних узлов в случае нетерминала). Так как класс будет
	содержать только данные и не обладать никакой сложной логикой, для упрощения разработки стоит использовать для его
	описания \texttt{dataclass} декоратор из библиотеки \texttt{dataclasses}.
	\item \textbf{PredictiveParsingTable} --- Таблица предсказывающего анализа. Она должна в себе для пар вида
	(Нетерминал, Терминал) хранить правило переписывания и функцию свертки, если таковая была задана. Очевидно,
	что класс таблицы должен обладать методами построения \texttt{Follow} множеств и построения самой таблицы.
	Оба метода, конечно же, должны будут вызываться в конструкторе таблицы. Также, для упрощения отладки, стоит
	реализовать метод для красивого текстового вывода таблицы.
	\item Дополнительные методы для класса \texttt{Parser}: класс нужно будет расширить методами для проведения ll1 разбора,
	построения таблицы и вычисления атрибутов токенов.
\end{itemize}

Также необходимо научить алгоритм работать с объектами этой библиотеки, классами для терминалов и нетерминалов, а ещё
научить алгоритм правильно обрабатывать и выбрасывать библиотечные ошибки, такие как \texttt{LexerError} и \texttt{ParseError}.

\subsubsection{Разработка разбора по алгоритму Эрли}

Принцип работы и особенности данного алгоритма разбора были описаны в разделе \ref{sec:earleyint}. Аналогично
алгоритму предсказывающего разбора, выделим основные компоненты в реализации и особенности, которые нужно будет учесть
при реализации.

В реализации данного алгоритма стоит выделить следующие компоненты:

\begin{itemize}
	\item Класс \texttt{EarleyState}, который представляет собой состояние в алгоритме Эрли. Он должен хранить в себе
	правило, по которому идет разбор, позицию в этом правиле, позицию в исходном коде, а также должен хранить
	координаты и атрибуты для завершенных состояний. Для удобства работы с классом стоит воспользоваться декоратором
	\texttt{dataclass} с опцией \texttt{(frozen=True)}, которая позволяет вычислять хэш для объектов класса.
	\item Также необходимо реализовать сам алгоритм разбора со всеми его вспомогательными методами, такими как
	\texttt{predict}, \texttt{scan}, \texttt{complete}. Возможно несколько вариантов реализации:
	добавить все новые методы в исходный класс или реализовать отдельный класс \texttt{EarleyParser}, который будет
	выполнять разбор.
\end{itemize}

Вне зависимости от выбранного пути реализации алгоритма, нужно будет не забыть добавить возможность вызывать разбор
методом Эрли из старого класса \texttt{Parser}.

\subsubsection{Поддержка ассоциативности и приоритета операций в стиле Bison}

Механизм задания ассоциативности и приоритета операторов, реализованный в Bison, позволяет интуитивно и
эффективно управлять группировкой выражений, избегая конфликтов при построении таблицы синтаксического анализа.

В реализации данной функциональности в библиотеке \texttt{parser\_edsl} целесообразно выделить следующие компоненты:

\begin{itemize}
	\item Для представления информации о приоритете и ассоциативности операторов можно добавить базовый класс
	\texttt{Precedence} и его наследники:
	\begin{itemize}
		\item \texttt{Left} для левой ассоциативности,
		\item \texttt{Right} для правой ассоциативности,
		\item \texttt{NonAssoc} для операторов без ассоциативности.
	\end{itemize}
	Каждый из этих классов будет хранить уровень приоритета (\texttt{level}) и строковое обозначение
	ассоциативности (\texttt{associativity}), что позволит в дальнейшем сравнивать приоритеты операторов.
	\item В классе \texttt{NonTerminal} необходимо будет дополнительно перегрузить оператор \verb|__ior__| для добавления
	возможности описания правил с операциями с ассоциативностью и приоритетом.
	\item В основном алгоритме разбора, описанном в классе \texttt{Parser}, при обнаружении конфликта
	shift/reduce будет необходимо:
	\begin{itemize}
		\item Из набора возможных действий выделять операция сдвига (\texttt{Shift}) и операция редукции
		(\texttt{Reduce}).
		\item Если для правила редукции задана информация о приоритете (объект \texttt{Precedence}),
		то сравнивать уровень приоритета токена, полученного от лексера, и уровень приоритета правила.
		\item Приоритет токена выше должен приводить к выполнению сдвига, а ниже --- к редукции.
		\item При равенстве приоритетов должна учитываться ассоциативность: для левых операторов выбирается редукция,
		 для правых --- сдвиг; отсутствие ассоциативности должно приводить к генерации ошибки.
	\end{itemize}
\end{itemize}

\subsection{Анализ существующей архитектуры и оценка изменений}

Перед внесением изменений необходимо тщательно изучить текущую архитектуру библиотеки \texttt{parser\_edsl}.
Планируется провести следующий анализ:
\begin{itemize}
	\item Оценка текущей модульности системы, что позволит в дальнейшем определить потенциальные места для внедрения новой
	функциональности и определить, что в библиотеке уже имеется, а что придется реализовывать.
	\item Оценка риска потенциального ухудшения качества работы библиотеки, как по корректности работы, так и по
	времени выполнения. Пользователь должен обладать возможностью легко использовать новые сценарии,
	но при этом важно сохранить обратную совместимость со старой
	версией библиотеки.
	\item Оценка наилучших мест в библиотеке для интеграции новой функциональности. Важно спланировать добавление
	новой функциональности так, чтобы код остался как можно более чистым и без ненужного дублирования каких-то
	фрагментов.
\end{itemize}

\subsubsection{Оценка модульности системы}

В процессе анализа исходного кода библиотеки было установлено, что структура проекта соответствует принципам
объектно-ориентированного программирования. Основные компоненты реализованы следующим образом:

\begin{itemize}
	\item Лексический анализ --- классы \texttt{Terminal}, \texttt{LiteralTerminal} и \texttt{ErrorTerminal} содержат всю логику,
	связанную с распознаванием лексем на основе регулярных выражений и точного сопоставления литералов.
	Эта часть системы достаточно изолирована, что позволяет в будущем добавлять обработку дополнительных исключительных
	ситуаций (например, дополнительные лексические ошибки) без глобальных изменений.
	\item Описание грамматики --- класс \texttt{NonTerminal} и связанные с ним механизмы (перегрузка оператора |= для
	добавления правил, использование семантических действий через \texttt{ExAction}) реализуют декларативное
	задание грамматики, предоставляя возможность добавлять новую функциональность не нарушая работу парсеров по правилам,
	написанным в старом формате.
	\item Класс \texttt{Parser} обладает методом \texttt{parse}, который умеет выполнять только LR(0) разбор.
	Однако, возможно есть способ переписать функцию так, чтобы она позволяла использовать разные методы разбора,
	При этом по умолчанию выполняя LR(0) разбор.
\end{itemize}

\subsubsection{Поиск наилучших мест для интеграции новой функциональности}

После анализа текущего кода библиотеки можно составить следующий план внедрения новой функциональности:

\begin{itemize}
	\item Для поддержки ошибок вычисления атрибутов токенов наиболее подходящим местом является расширение классов,
	отвечающих за вычисление атрибутов токенов (например, метод \texttt{match} в классе \texttt{Terminal}).
	Добавление обработки исключительных ситуаций в классе \texttt{Terminal} позволит обойтись изменениями в
	пределах лексического анализатора, не затрагивая модуль синтаксического разбора.
	\item Класс \texttt{Parser} является оптимальным местом для добавления новых методов разбора. Можно не реализовать
	дополнительные классы Parser, усложняя код. Важно сохранить общий интерфейс для парсера, чтобы
	пользователи могли выбирать нужный режим без изменения существующего кода.
	\item Добавление приоритета и ассоциативности операций потребует доработки перегрузки оператора \texttt{|=}
	у класса \texttt{NonTerminal}, а также в алгоритме разбора нужно будет учитывать ассоциативность операций при
	shift-reduce конфликтах.
\end{itemize}

\subsection{Разработка тестирования библиотеки}

Тщательное тестирование является неотъемлемой частью процесса разработки библиотеки, обеспечивая её надёжность,
корректность и устойчивость при внесении изменений. В рамках разработки тестирования для \texttt{parser\_edsl}
планируется реализация следующих уровней проверки:

\begin{itemize}
	\item Написание unit тестов. Unit тесты позволяют тестировать различные сценарии:
		\begin{itemize}
			\item Проверка, что уже написанный код работает, также, как и до этого
			\item Написание тестов на ещё не реализованные части библиотеки, чтобы ускорить процесс реализации
			благодаря автоматизированному тестированию.
		\end{itemize}
	\item Все unit тесты будут встроены в процесс непрерывной интеграции (CI), что позволит автоматически
	запускать набор тестов при каждом изменении кода. Это обеспечит своевременное обнаружение ошибок и
	повысит качество конечного кода.
\end{itemize}

Тестами стоит покрыть уже существующий алгоритм LR(0) разбора, а так же написать тесты под каждую новую функциональность,
добавляемую в библиотеку.

\section{Реализация расширений для библиотеки}

Для выполнения поставленной задачи необходимо реализовать возможность выбрасывания ошибок вычисления атрибутов,
новые алгоритмы разбора, а также ассоциативность и приоритет операций.

\subsection{Выбор инструментов разработки}

Вся разработка велась на языке Python, так как на нем реализована библиотека \texttt{parser\_edsl}, а переписывание
исходной библиотеки на другой язык нецелесообразно. Для разработки использовались следующие, встроенные в
язык библиотеки:

\begin{itemize}
	\item abc --- библиотека предоставляет возможность создания абстрактных базовых классов. Это позволяет
	создавать интерфейсы с обязательной реализацией определённых методов в
	наследниках, что помогает структурировать и стандартизировать код.
	\item collections --- библиотека содержит специализированные типы данных контейнеров, такие как
	\texttt{namedtuple}, \texttt{deque}, \texttt{Counter} и другие. Она упрощает работу с коллекциями и
	предоставляет удобные инструменты для организации данных.
	\item dataclasses --- библиотека облегчает создание классов, предназначенных для хранения данных. С помощью
	декоратора \texttt{@dataclass} автоматически генерируются методы \texttt{\_\_init\_\_}, \texttt{\_\_repr\_\_},
	\texttt{\_\_eq\_\_} и другие, что сокращает количество шаблонного кода и повышает читаемость.
\end{itemize}

Для тестирования проекта использовалась библиотека pytest, обладающая всей необходимой функциональностью, включая
возможность проверять, что программа выбрасывает конкретные типы исключений. Чтобы автоматизировать тестирование,
был настроен автозапуск тестов в Github Actions.

\subsection{Реализация расширенной функциональности}

Данный раздел содержит подробное описание изменений в библиотеке, связанных с добавлением новой функциональности.

\subsubsection{Реализация поддержки ошибки вычисления атрибутов}

В результате реализации был добавлен новый тип исключения \texttt{TokenAttributeError}, который наследуется от
базового класса ошибок библиотеки.

При вызове метода \texttt{match} класса \texttt{Terminal} происходит попытка вычисления атрибута с помощью
заданной функции. Если функция генерирует исключение \texttt{TokenAttributeError}, оно перехватывается,
и вызывается функция \texttt{pos\_from\_offset} для вычисления текущей позиции в исходном тексте.
Затем генерируется исключение \texttt{LexerError} с информацией о месте возникновения проблемы и с
соответствующим сообщением. Данный механизм позволяет локализовать и корректно обрабатывать ошибки на этапе
лексического анализа, а в дальнейшем они транслируются в синтаксические ошибки, что облегчает их перехват
и диагностику. Детали реализации представлены на листинге \ref{lst:error_impl}.

\subsubsection{Реализация предсказывающего разбора}

Для реализации предсказывающего разбора (LL(1)) в библиотеке разработан отдельный механизм, основанный на
построении таблицы предсказывающего разбора.

Для грамматики создаётся экземпляр класса \texttt{PredictiveParsingTable}, который вычисляет follow-множества
для всех нетерминалов и заполняет таблицу разбора. При возникновении конфликтов (если для одного и того же
нетерминала и терминала возможно более одного правила) генерируется исключение
\texttt{PredictiveTableConflictError}. Детали реализации представлены на листингах \ref{lst:ll1_impl_1}
- \ref{lst:ll1_impl_3}.

Метод \texttt{parse\_ll1} класса \texttt{Parser} реализует предсказывающий разбор с использованием стека,
содержащего узлы синтаксического дерева (типа \texttt{ParseTreeNode}). При обработке входного потока токенов
алгоритм поочерёдно извлекает символ верхушки стека:
\begin{itemize}
	\item Если символ является терминалом, он сравнивается с текущим токеном. При совпадении токен добавляется
	к узлу, и алгоритм переходит к следующему токену.
	\item Если символ является нетерминалом, производится обращение к LL(1) таблице для выбора соответствующей
	продукции, после чего нетерминал заменяется на правую часть выбранного правила.
\end{itemize}
После завершения разбора синтаксическое дерево обходится с помощью рекурсивного обхода
(методом \texttt{\_evaluate\_parse\_tree}), что позволяет вычислить итоговый семантический атрибут.

\subsubsection{Реализация разбора по алгоритму Эрли}

Для поддержки более сложных и неоднозначных грамматик реализован разбор по алгоритму Эрли.

Класс \texttt{EarleyState} представляет отдельное состояние алгоритма. Каждый объект этого класса хранит
правило разбора, позицию точки (dot), начальную и конечную позиции в потоке токенов, а также
накопленные атрибуты и координаты. Благодаря декоратору \texttt{@dataclass(frozen=True)} объекты состояний
могут быть использованы в качестве элементов множества. Детали реализации представлены на листингах
\ref{lst:earley_impl_1} --- \ref{lst:earley_impl_3}.

Класс \texttt{EarleyParser} реализует три ключевые операции:
\begin{itemize}
	\item \texttt{predict} --- предсказывает возможные правила для нетерминалов, ожидаемых в текущем состоянии.
	\item \texttt{scan} --- осуществляет сопоставление терминальных символов с текущим токеном.
	\item \texttt{complete} --- завершает обработку состояний, если правило разобрано
	полностью, и добавляет полученные результаты в предшествующие состояния.
\end{itemize}
Эти операции используются для поэтапного построения \emph{chart} (таблицы состояний), где на
каждой позиции входного потока фиксируются все возможные состояния разбора. Итоговая семантика
возвращается, если найдено единственное корректное состояние, начинающееся с начального символа.

\subsubsection{Реализация поддержки ассоциативности и приоритета операций в стиле Bison}

Для корректного разрешения конфликтов при разборе выражений реализован механизм задания ассоциативности и
приоритета операций, аналогичный функциональности, предлагаемой Bison.

Введён базовый класс \texttt{Precedence} и его наследники: \texttt{Left}, \texttt{Right} и \texttt{NonAssoc}.
Каждый из этих классов хранит уровень приоритета (\texttt{level}) и тип ассоциативности (\texttt{associativity}),
что позволяет однозначно определить порядок применения операторов. Детали реализации представлены на листинге
\ref{lst:bison_impl}.

В классе \texttt{NonTerminal} изменена перегрузка оператора \texttt{\_\_ior\_\_} для добавления правил
грамматики с указанием приоритета операцией.
Если правило передаётся в виде кортежа, где предпоследний элемент является объектом типа \texttt{Precedence},
то для соответствующего терминала устанавливается приоритет, а информация о приоритете сохраняется вместе с
продукцией. Это позволяет в дальнейшем использовать эти данные при разрешении конфликтов.

В методе \texttt{parse} класса \texttt{Parser} при обнаружении конфликта между операциями сдвига и редукции
происходит следующее:
\begin{itemize}
	\item Из набора возможных действий выделяются операция сдвига и операция редукции.
	\item Если для продукции редукции задана информация о приоритете, сравниваются приоритеты токена и
	продукции. Приоритет токена, вычисляемый по значению \texttt{priority} терминала, используется для
	определения, какую операцию выполнить.
	\item При равенстве уровней приоритета используется ассоциативность: для левых операторов выбирается
	редукция, для правых --- сдвиг; при отсутствии ассоциативности генерируется ошибка.
\end{itemize}


\subsection{Руководство пользователя}

\textbf{1. Установка и подключение библиотеки}

Библиотека написана на Python и не требует сложной установки. Для начала работы достаточно добавить файлы
библиотеки в проект и импортировать необходимые классы и функции, как представлено на листинге \ref{lst:guide1}.

\begin{listing}[H]
	\caption{Подключение библиотеки}
	\label{lst:guide1}
	\inputminted[style=bw, frame=single,fontsize = \footnotesize, linenos=false, xleftmargin = 1.5em]{python}{./listings/guide/1.py}
\end{listing}

\textbf{2. Определение лексических доменов}

Лексический анализ основан на описании терминалов. Для этого используются классы \texttt{Terminal} и
\texttt{LiteralTerminal}. Можно задать регулярное выражение и функцию преобразования для каждого терминала.
Пример представлен на листинге \ref{lst:guide2}. Также имеется возможность задавать приоритет терминалов.

\begin{listing}[H]
	\caption{Описание терминалов}
	\label{lst:guide2}
	\inputminted[style=bw, frame=single,fontsize = \footnotesize, linenos=false, xleftmargin = 1.5em]{python}{./listings/guide/2.py}
\end{listing}

\textbf{3. Описание грамматики}

Грамматика задаётся через объекты класса \texttt{NonTerminal}. Правила грамматики добавляются с помощью
оператора \verb||=|, что позволяет декларативно описывать продукции. При необходимости можно включить
семантические действия и информацию о приоритете операций, используя объекты классов
\texttt{Left}, \texttt{Right} или \texttt{NonAssoc}. Пример представлен на листинге \ref{lst:guide3}.

\begin{listing}[H]
	\caption{Описание нетерминалов}
	\label{lst:guide3}
	\inputminted[style=bw, frame=single,fontsize = \footnotesize, linenos=false, xleftmargin = 1.5em]{python}{./listings/guide/3.py}
\end{listing}

\textbf{4. Инициализация парсера и выбор алгоритма разбора}

После определения грамматики создаётся объект класса \texttt{Parser}. По умолчанию используется алгоритм LALR(1),
однако библиотека поддерживает и альтернативные режимы, такие как предсказывающий разбор (LL(1)) и
алгоритм Эрли. Для использования альтернативных алгоритмов доступны методы \texttt{parse\_ll1} и
\texttt{parse\_earley}. Пример представлен на листинге \ref{lst:guide4}.

\begin{listing}[H]
	\caption{Создание парсера и выполнение разбора}
	\label{lst:guide4}
	\inputminted[style=bw, frame=single,fontsize = \footnotesize, linenos=false, xleftmargin = 1.5em]{python}{./listings/guide/4.py}
\end{listing}

\textbf{5. Обработка ошибок}

При разборе возможны ошибки лексического или синтаксического анализа. Библиотека генерирует
исключения типа \texttt{ParseError} с подробной информацией о позиции ошибки и ожидаемых символах.
Это позволяет быстро находить и устранять проблемы в описании грамматики или входном тексте.

\textbf{6. Тестирование и отладка}

Для проверки корректности работы парсера рекомендуется использовать встроенные функции для печати таблицы разбора
(\texttt{print\_table}) и предсказывающей таблицы (\texttt{stringify\_ll1\_table}).

\section{Тестирование}

\subsection{Подготовка тестов}

Для тестирования библиотеки были подготовлены следующие наборы unit тестов:

\begin{itemize}
	\item Тесты для LALR разбора.
	\item Тесты для выбрасывания \texttt{TokenAttributeError} исключений
	\item Тесты для разбора по алгоритму предсказывающего анализатора.
	\item Тесты для алгоритма Эрли
	\item Тесты на арифметические выражения с разной ассоциативностью и различным приоритетом.
\end{itemize}

\subsection{Проведение тестирования}

Тестирование проводилось путем запуска pytest и проверки на успешное выполнение всех тестов.
Чтобы процесс тестирования был более прозрачным и автоматическим, их запуск был добавлен в Github Actions репозитория
с курсовой работой. Конфигурация github actions представлена на листинге \ref{lst:ghact}.

\begin{listing}[H]
	\caption{Создание парсера и выполнение разбора}
	\label{lst:ghact}
	\inputminted[style=bw, frame=single,fontsize = \footnotesize, linenos=false, xleftmargin = 1.5em]{python}{./listings/a.yaml}
\end{listing}

В результате удалось добиться успешного выполнения 100\% тестов.

\anonsection{Заключение}

В ходе выполнения данной курсовой работы была проведен анализ и оценка кодовой базы и архитектуры библиотеки
\texttt{parser\_edsl}, что позволило разработать добавление расширенной функциональности в библиотеку и в последствии
успешно реализовать её.

В результате работы были реализованы следующие расширения функциональности:

\begin{itemize}
	\item Добавлен механизм перехвата и обработки исключений, возникающих при вычислении атрибутов токенов,
	что позволяет обнаруживать ошибки на этапе лексического анализа и улучшает диагностику проблем.
	\item Разработан и интегрирован модуль, позволяющий строить предсказывающую таблицу и выполнять разбор
	с использованием алгоритма предсказывающего разбора. Это обеспечивает возможность эффективной работы с
	грамматиками, удовлетворяющими требованию LL(1), и повышает гибкость библиотеки.
	\item Внедрение алгоритма Эрли позволило расширить область применения библиотеки,
	обеспечив поддержку более сложных и неоднозначных грамматик.
	\item Реализован механизм задания приоритетов и ассоциативности операторов, что позволяет автоматически
	разрешать конфликтные ситуации типа shift/reduce и корректно обрабатывать выражения,
	содержащие операторы с различными уровнями приоритета.
\end{itemize}

Несмотря на достигнутые успехи, остаётся ряд направлений для дальнейшего улучшения библиотеки, включая оптимизацию
алгоритмов, добавление других алгоритмов разбора, расширение тестового покрытия и добавление взаимодействия
с другими инструментами анализа. В целом, проделанная работа является важным шагом в развитии инструментов
для конструирования компиляторов и может служить основой для будущих исследований и улучшений в данной области.

\cite{Golang}

\renewcommand\refname{СПИСОК ИСПОЛЬЗУЕМЫХ ИСТОЧНИКОВ}
% Список литературы
\clearpage
%\bibliographystyle{ugost2008s}  %utf8gost71u.bst} %utf8gost705u} %gost2008s}
{\catcode`"\active\def"{\relax}
\addcontentsline{toc}{section}{\protect\numberline{}\refname}%
%\bibliography{biblio} %здесь ничего не меняем, кроме, возможно, имени bib-файла
\printbibliography
}

\newpage

\anonsection{ПРИЛОЖЕНИЕ А}
\vspace{-30pt}

\begin{listing}[H]
	\caption{Реализация поддержки ошибки вычисления токенов.}
	\label{lst:error_impl}
	\inputminted[style=bw, frame=single,fontsize = \footnotesize, linenos=false, xleftmargin = 1.5em]{python}{./listings/error.py}
\end{listing}

\begin{listing}[H]
	\caption{Реализация алгоритма предсказывающего разбора.}
	\label{lst:ll1_impl_1}
	\inputminted[style=bw, frame=single,fontsize = \footnotesize, linenos=false, xleftmargin = 1.5em]{python}{./listings/ll1_1.py}
\end{listing}

\begin{listing}[H]
	\caption{Класс PredictiveParsingTable.}
	\label{lst:ll1_impl_2}
	\inputminted[style=bw, frame=single,fontsize = \footnotesize, linenos=false, xleftmargin = 1.5em]{python}{./listings/ll1_2.py}
\end{listing}

\begin{listing}[H]
	\caption{Класс PredictiveParsingTable. Продолжение.}
	\label{lst:ll1_impl_3}
	\inputminted[style=bw, frame=single,fontsize = \footnotesize, linenos=false, xleftmargin = 1.5em]{python}{./listings/ll1_3.py}
\end{listing}

\begin{listing}[H]
	\caption{Реализация алгоритма Эрли.}
	\label{lst:earley_impl_1}
	\inputminted[style=bw, frame=single,fontsize = \footnotesize, linenos=false, xleftmargin = 1.5em]{python}{./listings/earley_1.py}
\end{listing}

\begin{listing}[H]
	\caption{Реализация алгоритма Эрли.}
	\label{lst:earley_impl_2}
	\inputminted[style=bw, frame=single,fontsize = \footnotesize, linenos=false, xleftmargin = 1.5em]{python}{./listings/earley_2.py}
\end{listing}

\begin{listing}[H]
	\caption{Реализация алгоритма Эрли.}
	\label{lst:earley_impl_3}
	\inputminted[style=bw, frame=single,fontsize = \footnotesize, linenos=false, xleftmargin = 1.5em]{python}{./listings/earley_3.py}
\end{listing}


\begin{listing}[H]
	\caption{Реализация ассоциативности и приоритета операций.}
	\label{lst:bison_impl}
	\inputminted[style=bw, frame=single,fontsize = \footnotesize, linenos=false, xleftmargin = 1.5em]{python}{./listings/bison.py}
\end{listing}

\end{document}